\section{Zahlensysteme}
\subsection{Stellenwertsysteme}
\begin{center}
    \begin{minipage}{0.65\linewidth}
        \begin{center}
            \begin{tabular}{c l}
                $Z$ & zu berechnende positive Zahl\\
                $b$ & Basis von $Z$\\
                $a_i$ & Koeffizient
            \end{tabular}
        \end{center}
    \end{minipage}
    \hfill
    \begin{minipage}{0.3\linewidth}
        \begin{equation*}
            Z = \sum_{i=0}^{n} a_i \cdot b^i
        \end{equation*}
    \end{minipage}
\end{center}
\begin{flushleft}
    \begin{tabular}{l c l}
        Dezimal & $10$ & $a_i \in \{0, 1, \dots, 9\}$\\
        Dual/Binär & $2$ & $a_i \in \{0, 1\}$\\
        Oktal & $8$ & $a_i \in \{0, 1, \dots, 7\}$\\
        Hexa & $16$ & $a_i \in \{0, 1, \dots, 9, A, B, C, D, E, F\}$\\
    \end{tabular}
\end{flushleft}
\begin{minipage}{0.3\linewidth}
	\subsubsection{Binärsystem}
	\begin{itemize}
		\item Index b
		\item In Java \texttt{0b}: \texttt{0b0010 0000}
	\end{itemize}
\end{minipage}
\hfill
\begin{minipage}{0.5\linewidth}
	\subsubsection{Hexadezimalsysem}
	\begin{itemize}
		\item Umfasst 16 Werte
		\item Index h
		\item In Java \texttt{0x}: \texttt{0xAF3C}
	\end{itemize}
\end{minipage}
\subsection{Umwandlung Zahlensysteme}
	Es gilt: $Z = a_n \cdot b^n ... + a_3 \cdot b^3 + a_2 \cdot b^2 + a_1 \cdot b^1 + a_0 \cdot b^0$ \\
	Wobei $Z$ den gegebenen Wert in einem Zahlensystem der Basis $b$ darstellt. \\
	Beispiel: $1000_d$ in hexadezimal
\subsubsection{Binär zu Dezimal}
\begin{center}
    \begin{tabular}{c|c|c|c|c|c|c|c}
        $2^7$ & $2^6$ & $2^5$ & $2^4$ & $2^3$ & $2^2$ & $2^1$ & $2^0$\\
        $128$ & $64$ & $32$ & $16$ & $8$ & $4$ & $2$ & $1$
    \end{tabular}
\end{center}
\begin{center}
    \begin{tabular}{c|c|c|c}
        $2^{-1}$ & $2^{-2}$ & $2^{-3}$ & $2^{-4}$\\
        $0.5$ & $0.25$ & $0.125$ & $0.0625$
    \end{tabular}
\end{center}

\subsubsection{Binär zu Hex}
\begin{center}
    \begin{tabular}{c c||c c||c c||c c}
        $0000$ & $0$ & $0100$ & $4$ & $1000$ & $8$ & $1100$ & $C$\\
        $0001$ & $1$ & $0101$ & $5$ & $1001$ & $9$ & $1101$ & $D$\\
        $0010$ & $2$ & $0110$ & $6$ & $1010$ & $A$ & $1110$ & $E$\\
        $0011$ & $3$ & $0111$ & $7$ & $1011$ & $B$ & $1111$ & $F$\\
    \end{tabular}
\end{center}

\begin{minipage}[t]{0.4\linewidth}
\subsubsection*{Horner-Schema}
	\begin{align*}
		1000 : 16 &= 62 \text{ Rest: }  8 \rightarrow 8\\
		62 : 16 &=  3 \text{ Rest: } 14 \rightarrow E\\
		3 : 16 &=  0 \text{ Rest: } 3 \rightarrow 3\\
		1000_10 &= 3E8_16
	\end{align*}

\subsubsection*{Zweierkomplement}
	\begin{align*}
		+2_d = &00000010_b\\
		\text{invertieren: } &11111101 \\
		\text{1 addieren: } &00000001 \\
		-2_d = &11111110_b
	\end{align*}
	
\subsubsection*{Einerkomplement}
	\begin{align*}
		0_d = &00000000_b\\
		\text{invertieren: } &11111111 \\
		\text{1 addieren: } &00000001 \\
		0_d = &00000000_b
	\end{align*}
\end{minipage}
\hfill
\begin{minipage}[t]{0.45\linewidth}
\subsubsection*{Kommastellen}
	\begin{align*}
		26.6875_d &= 26_d + 0.6876_d \\
		\\
		26_d : 2 &= 13 \text{ Rest: } 0 \\
		13_d : 2 &= 6 \text{ Rest: } 1 \\
		6_d : 2 &= 0 \text{ Rest: } 0 \\
		3_d : 2 &= 0 \text{ Rest: } 1 \\
		1_d : 2 &= 0 \text{ Rest: } 1 \\
		\\
		0.6875_d \cdot 2 &= 0.3750 + 1 \\
		0.3750_d \cdot 2 &= 0.7500 + 0 \\
		0.7500_d \cdot 2 &= 0.5000 + 1 \\
		0.5000_d \cdot 2 &= 0.0000 + 1 \\
		\\
		26.6875_d &= 11010.1011_b
	\end{align*}
\end{minipage}

\subsection{Endliche Zahlen}
\begin{tabular}{| c | c | c | c |}
	\hline
	Reg. & Bez. & Unsinged & Int \\
	\hline
	\hline
	4 Bit & Nibble & $0 ... 15_d$ & $-8 ... +7$ \\
	\hline
	8 Bit & Byte & $255_d$ & $-128$/$127$ \\
	\hline
	16 Bit & Word & $65535_d$ & $-32768$/$32767$ \\
	\hline
	32 Bit & d. Word & $4.29_d \cdot 10^9$ & $\pm 2.15 \cdot 10^9$ \\
	\hline
	64 Bit & l. Word & $1.84_d \cdot 10^{19}$ & $\pm 9.22 \cdot 10^{18}$ \\
	\hline
	128 Bit & d.l. Word & $3.40_d \cdot 10^{38}$ & $\pm 1.70 \cdot 10^{38}$ \\
	\hline
\end{tabular}

\subsection{Carry und Overflow}
\begin{itemize}
	\item \textit{Carry} bezeichnet übertrag bei Operation die die Grösse des Registers überschreitet.
	\item Übertrag nach unten wird als \textit{borrow} bezeichnet.
	\item Bei vorzeichenbehafteten Zahlen tritt der Sprung dort auf, wo MSB wechselt.
	\item Bei MSB = 0 ist Registerinhalt null oder positiv. Bei MSB = 1 ist die Zahl negativ.
	\item \textit{Overflow} = vorzeichenbehafteter Übertrag.
	\item Für Carry, Borrow und Overflow existiert ein Flag im Mikroprozessor.
\end{itemize}

\subsection{Addition und Subtraktion}
\begin{enumerate}
	\item Summanden untereinander auflisten.
	\item Summanden bitweise addieren, dabei Übertrag beachten.
	\item Bei der Subtraktion das Zweierkomplement bilden und schliesslich addieren.
	\item Übertrag von MSB ignorieren.
\end{enumerate}

\subsection{Multiplikation}
\begin{center}
    \begin{minipage}[t]{0.70\linewidth}
        \begin{enumerate}
            \item Bitweise Multiplikation des Multiplikanden $a$ mit $b_i$ des Multiplikator.
            \item Sukzessive Multiplikationen werden um ein Bit ($0$) nach links verschoben.
            \item Anzahl Nachkommabits ergibt sich aus der Summe der Anzahl Nachkommabits der Operatoren.
        \end{enumerate}
    \end{minipage}
    \hfill
    \begin{minipage}[t]{0.25\linewidth}
        \begin{align*}
            b_0 \cdot a&\\
            +b_1 \cdot a~0\\
            +b_2 \cdot a~0~0&\\
            +b_3 \cdot a~0~0~0&\\
            \hline
            =\text{Sum}&
        \end{align*}
    \end{minipage}
\end{center}
